\documentclass{article}
\usepackage{amsmath,amsthm,amssymb}
\newcommand\numberthis{\addtocounter{equation}{1}\tag{\theequation}}
\usepackage{color}
\usepackage[letterpaper, margin=3.8cm]{geometry}
\definecolor{gray}{gray}{0.4}
\newtheorem*{theorem}{Theorem}
\newtheorem{lemma}{Lemma}
\theoremstyle{definition}
\newtheorem{definition}{Definition}

\title{My paper title}
\author{My name}
\date{The date}

\begin{document}
\maketitle

Let $\mathbb{B} = \{\lozenge, \blacklozenge \}$ where $\lozenge \neq \blacklozenge$. Let $T_{14}$ be a set and $C_{a_{\mbox{\tiny 1}}} : \mathbb{B} \to T_{14}$ and $C_{b_{\mbox{\tiny 1}}} : T_{14} \times 
                                  \mathbb{B} \to T_{14}$ functions such that all of the following conditions are met: 
\begin{itemize}
\item $C_{a_{\mbox{\tiny 1}}}$ and $C_{b_{\mbox{\tiny 1}}}$ are injective, 
\item for every $x_{\mbox{\tiny 21}}$, $x_{\mbox{\tiny 22}}$ and $x_{\mbox{\tiny 1}}$, we have $C_{a_{\mbox{\tiny 1}}}(x_{\mbox{\tiny 1}}) \neq C_{b_{\mbox{\tiny 1}}}(x_{\mbox{\tiny 21}}, x_{\mbox{\tiny 22}})$,
\item $T_{14}$  is \textit{covered} by $C_{a_{\mbox{\tiny 1}}}$ and $C_{b_{\mbox{\tiny 1}}}$ (i.e., $\mbox{\it image}(C_{a_{\mbox{\tiny 1}}})\cup\mbox{\it image}(C_{b_{\mbox{\tiny 1}}}) = T_{14}$)\footnote{This fact allows us to prove theorems about all the elements of $T_{14}$ by induction over the structure given by $C_{a_{\mbox{\tiny 1}}}$ and $C_{b_{\mbox{\tiny 1}}}$.}.
\end{itemize}

\noindent Then we proceed to define a function over this set.

\begin{definition}
Let $f_{\beta} : T_{14} \times  \mathbb{N} \to \mathbb{N}$ be the recursive function determined by the following equations (for any $a$, $b$ and $c$): 
\begin{align}
f_{\beta}(C_{a_{\mbox{\tiny 1}}}(a), b) &= b \label{f_2.simps_1}\\
f_{\beta}(C_{b_{\mbox{\tiny 1}}}(a, b), c) &= f_{\beta}(a, Suc(c)) \label{f_2.simps_2}
\end{align}
\end{definition}

\section{The Lemmas}
We begin by proving some necessary lemmas.
\begin{lemma}\label{lem_{n}}
For every $n \in T_{14}$, $p \in T_{14}$ and $q \in \mathbb{N}$
\[f_{\beta}(n, Suc(f_{\beta}(p, q))) = f_{\beta}(n, f_{\beta}(p, Suc(q)))\]
\end{lemma}
\begin{proof}
%[0, 1, 3, 0] 
              % (Induction on: p ) Current goals: [o, p], top goal: lem_{n}
              We proceed by induction on $p$. 
              
%[0, 1, 3, 0, 1] 
                 % (Start base case) Current goals: [o], top goal: lem_{n}
                 \vskip 1em
                 \noindent For the \textbf{base of induction} we need to prove the following statement:
                 \begin{align*}
                 &\forall\hspace{1pt} x \in \mathbb{B}. \; \forall\hspace{1pt} o \in T_{14}. \; \forall\hspace{1pt} r \in \mathbb{N}. \; \; f_{\beta}(o, Suc(f_{\beta}(C_{a_{\mbox{\tiny 1}}}(x), r))) = f_{\beta}(o, f_{\beta}(C_{a_{\mbox{\tiny 1}}}(x), Suc(r)))
                 \end{align*}
%[0, 1, 3, 0, 1, 0] 
                    % (Base case by simplification) Current goals: [], top goal: lem_{n}
                    This follows trivially from our definitions. \\
                    
%[0, 1, 3, 0, 2] 
                 \noindent For the \textbf{step of induction} we need to show that for every $r \in T_{14}$, $x_{\mbox{\tiny 2}} \in \mathbb{B}$, $o \in T_{14}$ and $s \in \mathbb{N}$, the inductive hypothesis (\ref{q}) entails the inductive goal (\ref{p}).
                 \begin{align}
                 \label{q} 
                 &\forall\hspace{1pt} t \in T_{14}. \; \forall\hspace{1pt} u \in \mathbb{N}. \; \; f_{\beta}(t, Suc(f_{\beta}(r, u))) = f_{\beta}(t, f_{\beta}(r, Suc(u)))
                 \tag{IH$_{\thesection}$}
                 \\
                 \label{p} \tag{IG$_{\thesection}$}
                 &f_{\beta}(o, Suc(f_{\beta}(C_{b_{\mbox{\tiny 1}}}(r, x_{\mbox{\tiny 2}}), s))) = f_{\beta}(o, f_{\beta}(C_{b_{\mbox{\tiny 1}}}(r, x_{\mbox{\tiny 2}}), Suc(s)))
                 \end{align}
                 We show this with the following chain of equalities: 
                 \begin{align*}
                 f_{\beta}(o, Suc(f_{\beta}(C_{b_{\mbox{\tiny 1}}}(r, x_{\mbox{\tiny 2}}), s)))
                 &= f_{\beta}(o, Suc(f_{\beta}(r, Suc(s)))) &\mbox{\color{gray}{by (\ref{f_2.simps_2})}}\\
                 &= f_{\beta}(o, f_{\beta}(r, Suc(Suc(s)))) &\mbox{\color{gray}{by (\ref{q})}}\\
                 &= f_{\beta}(o, f_{\beta}(C_{b_{\mbox{\tiny 1}}}(r, x_{\mbox{\tiny 2}}), Suc(s))) &\mbox{\color{gray}{by (\ref{f_2.simps_2})}}
                 \end{align*}
                 
Thus we conclude the proof of this lemma.
\end{proof}
\section{The Theorem}
In this section we prove the main result of this article.
\begin{theorem}\label{g1}
For every $a \in T_{14}$, $b \in T_{14}$ and $c \in \mathbb{N}$
\[f_{\beta}(a, f_{\beta}(b, c)) = f_{\beta}(b, f_{\beta}(a, c))\]
\end{theorem}
\begin{proof}
%[0] 
     % (Induction on: a ) Current goals: [i, j], top goal: g1
     We proceed by induction on $a$. 
     
%[0, 0] 
        % (Start base case) Current goals: [i], top goal: g1
        \vskip 1em
        \noindent For the \textbf{base of induction} we need to prove the following statement:
        \begin{align*}
        &\forall\hspace{1pt} x \in \mathbb{B}. \; \forall\hspace{1pt} d \in T_{14}. \; \forall\hspace{1pt} e \in \mathbb{N}. \; \; f_{\beta}(C_{a_{\mbox{\tiny 1}}}(x), f_{\beta}(d, e)) = f_{\beta}(d, f_{\beta}(C_{a_{\mbox{\tiny 1}}}(x), e))
        \end{align*}
%[0, 0, 0] 
           % (Base case by simplification) Current goals: [], top goal: g1
           This follows trivially from our definitions. \\
           
%[0, 1] 
        \noindent For the \textbf{step of induction} we need to show that for every $d \in T_{14}$, $x_{\mbox{\tiny 2}} \in \mathbb{B}$, $e \in T_{14}$ and $f \in \mathbb{N}$, the inductive hypothesis (\ref{k}) entails the inductive goal (\ref{j}).
        \begin{align}
        \label{k} 
        &\forall\hspace{1pt} g \in T_{14}. \; \forall\hspace{1pt} h \in \mathbb{N}. \; \; f_{\beta}(d, f_{\beta}(g, h)) = f_{\beta}(g, f_{\beta}(d, h))
        \tag{IH$_{\thesection}$}
        \\
        \label{j} \tag{IG$_{\thesection}$}
        &f_{\beta}(C_{b_{\mbox{\tiny 1}}}(d, x_{\mbox{\tiny 2}}), f_{\beta}(e, f)) = f_{\beta}(e, f_{\beta}(C_{b_{\mbox{\tiny 1}}}(d, x_{\mbox{\tiny 2}}), f))
        \end{align}
        We show this with the following chain of equalities: 
        \begin{align*}
        f_{\beta}(C_{b_{\mbox{\tiny 1}}}(d, x_{\mbox{\tiny 2}}), f_{\beta}(e, f))
        &= f_{\beta}(d, Suc(f_{\beta}(e, f))) &\mbox{\color{gray}{by (\ref{f_2.simps_2})}}\\
        &= f_{\beta}(d, f_{\beta}(e, Suc(f))) &\mbox{\color{gray}{by Lemma \ref{lem_{n}}}}\\
        &= f_{\beta}(e, f_{\beta}(d, Suc(f))) &\mbox{\color{gray}{by (\ref{k})}}\\
        &= f_{\beta}(e, f_{\beta}(C_{b_{\mbox{\tiny 1}}}(d, x_{\mbox{\tiny 2}}), f)) &\mbox{\color{gray}{by (\ref{f_2.simps_2})}}
        \end{align*}
        
Thus we conclude the proof of this theorem.
\end{proof}

\end{document} 