\documentclass{article}
\usepackage{amsmath,amsthm,amssymb}
\newcommand\numberthis{\addtocounter{equation}{1}\tag{\theequation}}
\usepackage{color}
\usepackage[letterpaper, margin=3.8cm]{geometry}
\definecolor{gray}{gray}{0.4}
\newtheorem*{theorem}{Theorem}
\newtheorem{lemma}{Lemma}
\theoremstyle{definition}
\newtheorem{definition}{Definition}

\title{My paper title}
\author{My name}
\date{The date}

\begin{document}
\maketitle

Let $\mathbb{B} = \{\lozenge, \blacklozenge \}$ where $\lozenge \neq \blacklozenge$. Let $\mathbb{N}$ be the set of natural numbers, with ${\tt Suc}$ the \textit{successor} function. Let $T_{2}$ be a set and $C_{c} : \mathbb{N} \times 
 \mathbb{N} \to T_{2}$ and $C_{d} : T_{2} \times  \mathbb{B} \to T_{2}$ functions such that all of the following conditions are met: 
\begin{itemize}
\item $C_{c}$ and $C_{d}$ are injective, 
\item for every $x_{\mbox{\tiny 21}}$, $x_{\mbox{\tiny 22}}$, $x_{\mbox{\tiny 11}}$ and $x_{\mbox{\tiny 12}}$, we have $C_{c}(x_{\mbox{\tiny 11}}, x_{\mbox{\tiny 12}}) \neq C_{d}(x_{\mbox{\tiny 21}}, x_{\mbox{\tiny 22}})$,
\item $T_{2}$  is \textit{covered} by $C_{c}$ and $C_{d}$ (i.e., $\mbox{\it image}(C_{c})\cup\mbox{\it image}(C_{d}) = T_{2}$)\footnote{This fact allows us to prove theorems about all the elements of $T_{2}$ by induction over the structure given by $C_{c}$ and $C_{d}$.}.
\end{itemize}

\noindent Then we can proceed to define a function over this set.

\begin{definition}
Let $f_{\alpha} : T_{2} \times  \mathbb{N} \to \mathbb{N}$ be the recursive function determined by the following equations (for any $a$, $b$ and $c$): 
\begin{align}
f_{\alpha}(C_{c}(a, b), c) &= c \label{f_1.simps_1}\\
f_{\alpha}(C_{d}(a, b), c) &= {\tt{Suc}}(f_{\alpha}(a, {\tt{Suc}}(f_{\alpha}(a, {\tt{Suc}}(c))))) \label{f_1.simps_2}
\end{align}
\end{definition}

\section{The Theorem}

\begin{theorem}\label{g1}
For every $a \in T_{2}$ and $b \in \mathbb{N}$
\[{\tt{Suc}}(f_{\alpha}(a, {\tt{Suc}}(b))) = f_{\alpha}(a, {\tt{Suc}}({\tt{Suc}}(b)))\]
\end{theorem}
\begin{proof}
%[0] 
     % (Induction on: a ) Current goals: [i, j], top goal: g1
     We proceed by induction on $a$. 
     
%[0, 0] 
        % (Start base case) Current goals: [i], top goal: g1
        \vskip 1em
        \noindent For the \textbf{base of induction} we need to prove the following statement:
        \begin{align*}
        &\forall\hspace{1pt} x_{\mbox{\tiny 1}} \in \mathbb{N}. \; \forall\hspace{1pt} x_{\mbox{\tiny 2}} \in \mathbb{N}. \; \forall\hspace{1pt} c \in \mathbb{N}. \; \; {\tt{Suc}}(f_{\alpha}(C_{c}(x_{\mbox{\tiny 1}}, x_{\mbox{\tiny 2}}), {\tt{Suc}}(c))) = f_{\alpha}(C_{c}(x_{\mbox{\tiny 1}}, x_{\mbox{\tiny 2}}), {\tt{Suc}}({\tt{Suc}}(c)))
        \end{align*}
%[0, 0, 0] 
           % (Base case by simplification) Current goals: [], top goal: g1
           This follows trivially from our definitions. \\
           
%[0, 1] 
        \noindent For the \textbf{step of induction} we need to show that for every $c \in T_{2}$, $x_{\mbox{\tiny 2}} \in \mathbb{B}$ and $d \in \mathbb{N}$, the inductive hypothesis (\ref{k}) entails the inductive goal (\ref{j}).
        \begin{align}
        \label{k} 
        &\forall\hspace{1pt} e \in \mathbb{N}. \; \; {\tt{Suc}}(f_{\alpha}(c, {\tt{Suc}}(e))) = f_{\alpha}(c, {\tt{Suc}}({\tt{Suc}}(e)))
        \tag{IH$_{\thesection}$}
        \\
        \label{j} \tag{IG$_{\thesection}$}
        &{\tt{Suc}}(f_{\alpha}(C_{d}(c, x_{\mbox{\tiny 2}}), {\tt{Suc}}(d))) = f_{\alpha}(C_{d}(c, x_{\mbox{\tiny 2}}), {\tt{Suc}}({\tt{Suc}}(d)))
        \end{align}
        First, notice that the following is true: 
        \begin{align*}
        {\tt{Suc}}(f_{\alpha}(c, {\tt{Suc}}(f_{\alpha}(c, {\tt{Suc}}({\tt{Suc}}(d))))))
        &= f_{\alpha}(c, {\tt{Suc}}({\tt{Suc}}(f_{\alpha}(c, {\tt{Suc}}({\tt{Suc}}(d)))))) &\mbox{\color{gray}{by (\ref{k})}}\\
        &= f_{\alpha}(c, {\tt{Suc}}(f_{\alpha}(c, {\tt{Suc}}({\tt{Suc}}({\tt{Suc}}(d)))))) &\mbox{\color{gray}{by (\ref{k})}}
        \numberthis \label{n}
        \end{align*}
        Ultimately, we can show IG$_{\thesection}$ as follows:
        \begin{align*}
        {\tt{Suc}}(f_{\alpha}(C_{d}(c, x_{\mbox{\tiny 2}}), {\tt{Suc}}(d)))
        &= {\tt{Suc}}({\tt{Suc}}(f_{\alpha}(c, {\tt{Suc}}(f_{\alpha}(c, {\tt{Suc}}({\tt{Suc}}(d))))))) &\mbox{\color{gray}{by (\ref{f_1.simps_2})}}\\
        &= {\tt{Suc}}(f_{\alpha}(c, {\tt{Suc}}(f_{\alpha}(c, {\tt{Suc}}({\tt{Suc}}({\tt{Suc}}(d))))))) &\mbox{\color{gray}{by (\ref{n})}}\\
        &= f_{\alpha}(C_{d}(c, x_{\mbox{\tiny 2}}), {\tt{Suc}}({\tt{Suc}}(d))) &\mbox{\color{gray}{by (\ref{f_1.simps_2})}}
        \end{align*}
        
Thus we conclude the proof of this theorem.
\end{proof}

\end{document} 