\documentclass{article}
\usepackage{amsmath,amsthm,amssymb}
\usepackage{color}
\usepackage[letterpaper, margin=4cm]{geometry}
\definecolor{gray}{gray}{0.4}
\newtheorem*{theorem}{Theorem}
\newtheorem{lemma}{Lemma}
\theoremstyle{definition}
\newtheorem{definition}{Definition}

\title{My paper title}
\author{My name}
\date{The date}

\begin{document}
\maketitle

Let $\mathbb{B} = \{\lozenge, \blacklozenge \}$ where $\lozenge \neq \blacklozenge$. Let $T$ be a set and $C_{a} : \mathbb{B} \times 
                                                 \mathbb{B} \to T$ and $C_{b} : T \to T$ functions such that all of the following conditions are met: 
\begin{itemize}
\item $C_{a}$ and $C_{b}$ are injective, 
\item for every $x_{\mbox{\tiny 2}}$, $x_{\mbox{\tiny 11}}$ and $x_{\mbox{\tiny 12}}$, we have $C_{a}(x_{\mbox{\tiny 11}}, x_{\mbox{\tiny 12}}) \neq C_{b}(x_{\mbox{\tiny 2}})$,
\item $T$ is \textit{covered} by $C_{a}$ and $C_{b}$ (i.e., $\mbox{\it image}(C_{a})\cup\mbox{\it image}(C_{b}) =T$).
\end{itemize}

\noindent Then we proceed to define a function over this set.

\begin{definition}
Let $f_{\alpha} : T \times  T \to T$ be the recursive function determined by the following equations (for any x, y and z): 
\begin{align}
f_{\alpha}(C_{a}(x, y), z) &= z \label{f_1.simps_1}\\
f_{\alpha}(C_{b}(x), y) &= C_{b}(f_{\alpha}(x, y)) \label{f_1.simps_2}
\end{align}
\end{definition}

\section{The Lemmas}
We begin by proving some necessary lemmas.
\begin{lemma}\label{lem_{n}}
For every $n \in T$ and $p \in T$
\[f_{\alpha}(n, C_{b}(p)) = C_{b}(f_{\alpha}(n, p))\]
\end{lemma}
\begin{proof}
%[0, 1, 3, 0] 
              % (Induction on: n ) Current goals: [o, p], top goal: lem_{n}
              We proceed by induction on $n$. 
              
%[0, 1, 3, 0, 1] 
                 % (Start base case) Current goals: [o], top goal: lem_{n}
                 \vskip 1em
                 \noindent For the \textbf{base of induction} we need to prove the following statement:
                 \begin{align*}
                 & \forall\, x_{\mbox{\tiny 1}} \in \mathbb{B}. \; \forall\, x_{\mbox{\tiny 2a}} \in \mathbb{B}. \; \forall\, q \in T. \; \notag \\
                 & \mbox{\quad\quad\quad\quad\quad}f_{\alpha}(C_{a}(x_{\mbox{\tiny 1}}, x_{\mbox{\tiny 2a}}), C_{b}(q)) = C_{b}(f_{\alpha}(C_{a}(x_{\mbox{\tiny 1}}, x_{\mbox{\tiny 2a}}), q))
                 \end{align*}
%[0, 1, 3, 0, 1, 0] 
                    % (Base case by simplification) Current goals: [], top goal: lem_{n}
                    This follows trivially from our definitions. \\
                    
%[0, 1, 3, 0, 2] 
                 \noindent For the \textbf{step of induction} we need to show that for every $o \in T$ and $q \in T$, the inductive hypothesis (\ref{q}) entails the goal (\ref{p}).
                 \begin{align}
                 \label{q}
                 \forall\, r \in T. \; \; f_{\alpha}(o, C_{b}(r)) = C_{b}(f_{\alpha}(o, r))
                 \\
                 \label{p}
                 f_{\alpha}(C_{b}(o), C_{b}(q)) = C_{b}(f_{\alpha}(C_{b}(o), q))
                 \end{align}
                 We show this with the following chain of equalities: 
                 \begin{align*}
                 f_{\alpha}(C_{b}(o), C_{b}(q))
                 &= C_{b}(f_{\alpha}(o, C_{b}(q))) &\mbox{\color{gray}{by (\ref{f_1.simps_2})}}\\
                 &= C_{b}(C_{b}(f_{\alpha}(o, q))) &\mbox{\color{gray}{by (\ref{q})}}\\
                 &= C_{b}(f_{\alpha}(C_{b}(o), q)) &\mbox{\color{gray}{by (\ref{f_1.simps_2})}}
                 \end{align*}
                 
Thus we conclude the proof of this lemma.
\end{proof}
\section{The Theorem}
In this section we prove the main result of this article
\begin{theorem}\label{g1}
For every $a \in T$, $b \in T$ and $c \in T$
\[f_{\alpha}(b, f_{\alpha}(a, c)) = f_{\alpha}(a, f_{\alpha}(b, c))\]
\end{theorem}
\begin{proof}
%[0] 
     % (Induction on: b ) Current goals: [i, j], top goal: g1
     We proceed by induction on $b$. 
     
%[0, 0] 
        % (Start base case) Current goals: [i], top goal: g1
        \vskip 1em
        \noindent For the \textbf{base of induction} we need to prove the following statement:
        \begin{align*}
        & \forall\, x_{\mbox{\tiny 1}} \in \mathbb{B}. \; \forall\, x_{\mbox{\tiny 2a}} \in \mathbb{B}. \; \forall\, d \in T. \; \forall\, e \in T. \; \notag \\
        & \mbox{\quad\quad\quad\quad\quad}f_{\alpha}(C_{a}(x_{\mbox{\tiny 1}}, x_{\mbox{\tiny 2a}}), f_{\alpha}(d, e)) = f_{\alpha}(d, f_{\alpha}(C_{a}(x_{\mbox{\tiny 1}}, x_{\mbox{\tiny 2a}}), e))
        \end{align*}
%[0, 0, 0] 
           % (Base case by simplification) Current goals: [], top goal: g1
           This follows trivially from our definitions. \\
           
%[0, 1] 
        \noindent For the \textbf{step of induction} we need to show that for every $d \in T$, $e \in T$ and $f \in T$, the inductive hypothesis (\ref{k}) entails the goal (\ref{j}).
        \begin{align}
        \label{k}
        \forall\, g \in T. \; \forall\, h \in T. \; \; f_{\alpha}(d, f_{\alpha}(g, h)) = f_{\alpha}(g, f_{\alpha}(d, h))
        \\
        \label{j}
        f_{\alpha}(C_{b}(d), f_{\alpha}(e, f)) = f_{\alpha}(e, f_{\alpha}(C_{b}(d), f))
        \end{align}
        We show this with the following chain of equalities: 
        \begin{align*}
        f_{\alpha}(C_{b}(d), f_{\alpha}(e, f))
        &= C_{b}(f_{\alpha}(d, f_{\alpha}(e, f))) &\mbox{\color{gray}{by (\ref{f_1.simps_2})}}\\
        &= C_{b}(f_{\alpha}(e, f_{\alpha}(d, f))) &\mbox{\color{gray}{by (\ref{k})}}\\
        &= f_{\alpha}(e, C_{b}(f_{\alpha}(d, f))) &\mbox{\color{gray}{by Lemma \ref{lem_{n}}}}\\
        &= f_{\alpha}(e, f_{\alpha}(C_{b}(d), f)) &\mbox{\color{gray}{by (\ref{f_1.simps_2})}}
        \end{align*}
        
Thus we conclude the proof of this theorem.
\end{proof}

\end{document} 
